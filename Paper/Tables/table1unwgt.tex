\begin{table}[ht]
\caption{Differences in outcomes by college major, relative to education majors}
\label{tab:samplemeans}
\centering
\begin{threeparttable}
\begin{tabular}{lcccccc}
\toprule
                                & Education & Soc Sci & Other   & Business & STEM     \\
\midrule
Log Earnings                    & 0.00      & 0.19   & 0.163   & 0.415   & 0.426    \\
                                & (---)      & (0.063)   & (0.052)   & (0.06)   & (0.065)    \\
Pr(Lives outside birth state)   & 0.00      & 0.12   & 0.129   & 0.077   & 0.132    \\
                                & (---)      & (0.058)   & (0.062)   & (0.05)   & (0.058)    \\
Pr(Works in related occupation) & 0.00      & -0.172   & -0.12   & -0.032   & -0.034    \\
                                & (---)      & (0.043)   & (0.046)   & (0.056)   & (0.054)    \\
\midrule
Frequency (\%)                  & 5.29   & 11.49   & 21.09   & 27.35   & 34.77 \\ 
N                               & 54,171 & 117,593 & 215,872 & 279,971 & 355,918 \\ 
\bottomrule
\end{tabular}
{\footnotesize {\raggedright Notes: Regression estimates at national level, controlling for demographics, advanced degree status, CBSA dummies, and a cubic in potential experience. Standard deviation of state-specific estimates reported below in parentheses. All variables except for log earnings and distance are expressed in percentage points and estimated from linear probability models. Sample taken from the 2010-2015 American Community Survey and is restricted to males ages 22-54 with a bachelor's degree or higher. Additional details on sample selection can be found in Table \ref{tab:sampleselection}.}}
\end{threeparttable}
\end{table}
