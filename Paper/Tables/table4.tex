\begin{table}[ht]
\caption{Percentage change in returns when correcting for selection}
\label{tab:aggpctreturn}
\centering
\begin{threeparttable}
\begin{tabular}{lcccccccc}
\toprule
                & \multicolumn{4}{c}{Unrelated occupation} & \multicolumn{4}{c}{Related occupation} \\
\cmidrule(lr{.5em}){2-5}\cmidrule(lr{.5em}){6-9}
Major           & p10   & Median & p90 & No. significant  & p10   & Median & p90 & No. significant  \\ 
\midrule
\emph{Bachelor's degrees}       &&&&&& \\
\qquad{}Education & 0.0 & 0.0 & 0.0 & 0 & 0.0 & 0.0 & 0.0 & 0    \\
\qquad{}Social Science & -39.7 & -3.0 & 34.6 & 1 & -4.3 & 2.0 & 9.0 & 1    \\
\qquad{}Other & -25.7 & -4.8 & 80.5 & 0 & -8.8 & 0.5 & 6.7 & 2    \\
\qquad{}Business & -12.1 & -5.0 & -1.7 & 4 & -6.5 & -0.8 & 0.4 & 1    \\
\qquad{}STEM & -14.4 & -4.6 & -2.2 & 4 & -8.8 & -1.8 & 0.0 & 2    \\
\emph{Advanced degrees}       &&&&&& \\
\qquad{}Education & -60.1 & -5.3 & 2.3 & 2 & -98.3 & -23.1 & 43.1 & 3    \\
\qquad{}Social Science & -29.9 & -10.1 & -3.3 & 6 & -61.1 & -31.5 & 6.3 & 4    \\
\qquad{}Other & -32.3 & -10.2 & -1.1 & 4 & -116.6 & -36.1 & 133.4 & 4    \\
\qquad{}Business & -24.2 & -7.0 & -0.1 & 4 & -28.1 & -16.4 & -5.0 & 6    \\
\qquad{}STEM & -20.8 & -5.6 & -2.4 & 4 & -28.1 & -14.8 & -4.9 & 6    \\
\bottomrule
\end{tabular}
{\footnotesize {\raggedright Notes: Summary statistics of the 15-location distribution of the percentage change between uncorrected and corrected returns to majors. ``No. significant'' counts the number of locations satisfying the following conditions: $(i)$ the Hausman test statistic overturns the null hypothesis of no difference between corrected and uncorrected at the 5\% level; $(ii)$ both the uncorrected and corrected coefficients are statistically different from zero at the 5\% level; $(iii)$ the percentage difference between the corrected and uncorrected returns is significantly different from zero at the 5\% level; and $(iv)$ the percentage difference exceeds 10\% in magnitude. Percentage changes are least informative for education, social science, and other majors because these majors have bases (i.e. uncorrected returns) that may be very close to zero.}}
\end{threeparttable}
\end{table}
